\documentclass{ctexart}

\usepackage{graphicx}
\usepackage{amsmath}

\title{Homework1:L'H\^opital's rule's describe and proof}

\author{Ang Chen Han \\ Information and Computational Science 3200300133}

\begin{document}

\maketitle


L'H\^opital's rule is a theorem which provides a technique to evalute limits of indeterminate forms.Application of the rule often converts an indeterminate form to an expression that can be easily evaluted by substitution.

\section{Question Describe}
Question describe as below:
\subsection{If function $f(x)$ and $g(x)$ in the area of $U(a,\delta)$ are derivable and satisfy :}
\begin{align}
\lim_{x \rightarrow a}f(x)&=0\\
\lim_{x \rightarrow a}g(x)&=0\\
\lim_{x \rightarrow a}\frac{f'(x)}{g'(x)}&=l \\
\text{then exist : }\lim_{x \rightarrow a}\frac{f(x)}{g(x)}&=l
\end{align}

\subsection{If function $f(x)$ and $g(x)$ in the area of $U(\infty,\delta)$ are derivable and satisfy :}
\begin{align}
\lim_{x \rightarrow \infty}f(x)&=0\label{eq::first}\\
\lim_{x \rightarrow \infty}g(x)&=0\label{eq::second}\\
\lim_{x \rightarrow \infty}\frac{f'(x)}{g'(x)}&=l \\
\text{then exist : }\lim_{x \rightarrow \infty}\frac{f(x)}{g(x)}&=l
\end{align}

\subsection{If function $f(x)$ and $g(x)$ in the area of $U(a,\delta)$ are derivable and satisfy :}
\begin{align}
\lim_{x \rightarrow a}g(x)&=\infty\\
\lim_{x \rightarrow a}\frac{f'(x)}{g'(x)}&=l \\
\text{then exist : }\lim_{x \rightarrow a}\frac{f(x)}{g(x)}&=l
\end{align}

\subsection{If function $f(x)$ and $g(x)$ in the area of $U(\infty,\delta)$ are derivable and satisfy :}
\begin{align}
\lim_{x \rightarrow \infty}g(x)&=\infty\label{eq::third}\\
\lim_{x \rightarrow \infty}\frac{f'(x)}{g'(x)}&=l \\
\text{then exist : }\lim_{x \rightarrow \infty}\frac{f(x)}{g(x)}&=l
\end{align}


\section{Proof the Rule}
Just proof the situation when~$\lim_{x \rightarrow a}\frac{f'(x)}{g'(x)}=l<\infty$ ,when $l=\infty$ ,both are same.

\subsection{First proof (\ref{eq::first}) and (\ref{eq::second})}
\begin{align}
&\text{Let }\lim_{x \rightarrow \infty}f(x)=\lim_{x \rightarrow \infty}g(x)=0\notag \\
&\forall\epsilon>0 ,\exists\delta>0 ,\text{when }x\in(a,a+\delta)\notag \\
&l-\epsilon<\frac{f'(x)}{g'(x)}<l+\epsilon ,\\
&\text{for }(x,x_0)\subset(a,a+\delta) ,\text{from Cauchy's mean value theorem ,got}\notag\\
&\xi\in(x,x_0),l-\epsilon<\frac{f(x)-f(x_0)}{g(x)-g(x_0)}=\frac{f'(\xi)}{g'(\xi)}<l+\epsilon\\
&\text{let }x_0 \rightarrow a ,1-\frac{g(x_0)}{g(x)} \rightarrow 1\\
&\text{so that ,}\limsup_{x \rightarrow a^+}\frac{f(x)}{g(x)}\le l+\epsilon \\
&\text{from arbitrariness of }\epsilon ,\limsup_{x \rightarrow a^+}\frac{f(x)}{g(x)} \le l \label{eq::left}\\
&\text{similarly ,}\liminf_{x \rightarrow a^+}\frac{f(x)}{g(x)} \ge l \label{eq::right}\\
&\text{from (\ref{eq::left}) and (\ref{eq::right}) ,}\lim_{x \rightarrow a^+}\frac{f(x)}{g(x)}=l 
\end{align}

\subsection{Now proof (\ref{eq::third})}
\begin{align}
\frac{f(x)}{g(x)}&=\frac{f(x)-f(x_0)}{g(x)}+\frac{f(x_0)}{g(x)}\notag \\
&=\frac{g(x)-g(x_0)}{g(x)}*\frac{f(x)-f(x_0)}{g(x)-g(x_0)}+\frac{f(x_0)}{g(x)}\notag\\
&=[1-\frac{g(x_0)}{g(x)}]\frac{f(x)-f(x_0)}{g(x)-g(x_0)}+\frac{f(x_0)}{g(x)}\\
\arrowvert\frac{f(x)}{g(x)}-l\arrowvert&=\arrowvert[1-\frac{g(x_0)}{g(x)}]\frac{f(x)-f(x_0)}{g(x)-g(x_0)}+\frac{f(x_0)}{g(x)}-l\arrowvert \notag \\
&\le\arrowvert1-\frac{g(x_0)}{g(x)}\arrowvert*\arrowvert\frac{f(x)-f(x_0)}{g(x)-g(x_0)}-l\arrowvert+\arrowvert\frac{f(x_0)-lg(x_0)}{g(x)}\arrowvert\label{eq::one}
\end{align}

\begin{align}
&\text{then ,}\arrowvert\frac{f(x)-f(x_0)}{g(x)-g(x_0)}-l\arrowvert=\arrowvert\frac{f'(\xi)}{g'(\xi)}-l\arrowvert<\epsilon \notag\\
&\text{another ,}\lim_{x \rightarrow a^+}g(x)=\infty ,\text{when x near a}\notag\\
&\arrowvert1-\frac{g(x_0)}{g(x)}\arrowvert<C,\forall C \in R\label{eq::two}\\
&\arrowvert\frac{f(x)-f(x_0)}{g(x)-g(x_0)}-l\arrowvert\rightarrow0\label{eq::three}\\
&\arrowvert\frac{f(x_0)-lg(x_0)}{g(x)}\arrowvert\rightarrow0\label{eq::four}\\
&\text{from (\ref{eq::one})(\ref{eq::two})(\ref{eq::three})(\ref{eq::four}) ,}\lim_{x \rightarrow a^+}\frac{f(x)}{g(x)}=l=\lim_{x \rightarrow a^+}\frac{f'(x)}{g'(x)}
\end{align}
\end{document}
